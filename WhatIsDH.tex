\documentclass[]{article}
\usepackage{lmodern}
\usepackage{amssymb,amsmath}
\usepackage{ifxetex,ifluatex}
\usepackage{fixltx2e} % provides \textsubscript
\ifnum 0\ifxetex 1\fi\ifluatex 1\fi=0 % if pdftex
  \usepackage[T1]{fontenc}
  \usepackage[utf8]{inputenc}
\else % if luatex or xelatex
  \ifxetex
    \usepackage{mathspec}
    \usepackage{xltxtra,xunicode}
  \else
    \usepackage{fontspec}
  \fi
  \defaultfontfeatures{Mapping=tex-text,Scale=MatchLowercase}
  \newcommand{\euro}{€}
\fi
% use upquote if available, for straight quotes in verbatim environments
\IfFileExists{upquote.sty}{\usepackage{upquote}}{}
% use microtype if available
\IfFileExists{microtype.sty}{%
\usepackage{microtype}
\UseMicrotypeSet[protrusion]{basicmath} % disable protrusion for tt fonts
}{}
\ifxetex
  \usepackage[setpagesize=false, % page size defined by xetex
              unicode=false, % unicode breaks when used with xetex
              xetex]{hyperref}
\else
  \usepackage[unicode=true]{hyperref}
\fi
\hypersetup{breaklinks=true,
            bookmarks=true,
            pdfauthor={Jordan Lewis, Amber Cantrell, Andrew Wyatt},
            pdftitle={What is Digital Humanities?},
            colorlinks=true,
            citecolor=blue,
            urlcolor=blue,
            linkcolor=magenta,
            pdfborder={0 0 0}}
\urlstyle{same}  % don't use monospace font for urls
\setlength{\parindent}{0pt}
\setlength{\parskip}{6pt plus 2pt minus 1pt}
\setlength{\emergencystretch}{3em}  % prevent overfull lines
\setcounter{secnumdepth}{0}

\title{What is Digital Humanities?}
\author{Jordan Lewis, Amber Cantrell, Andrew Wyatt}
\date{February 3, 2015}

\begin{document}
\maketitle

\section{What Is DH?}\label{what-is-dh}

\subsubsection{Digital Humanities?}\label{digital-humanities}

When starting off, I had no idea what Digital humanities entailed. I
knew there would some working with computers, but didn't realize how
much computer work would be needed. I am far from a computer literate
person, so, this is quite the struggle for me. When finding out just how
many programs I needed for this class, I was confused as to why. I still
didn't quite understand what Digital Humanities is.

\subsubsection{Discovering DH}\label{discovering-dh}

Upon reading required texts, I finally had a definition for Digital
Humanities. In an article by Kirschenbaum, \emph{What Is Digital
Humanities and What is it Doing in English Departments}, this is the
definition that was given:

\begin{quote}
``The digital humanities, also known as humanities computing, is a field
of study, research, teaching, and invention concerned with the
intersection of computing and the disciplines of the humanities.''
(Kirschenbaum 2)
\end{quote}

\subsubsection{What Does That Mean?}\label{what-does-that-mean}

For starters, you know that there is the word `computing' in the
definition, meaning there is involvement of computers and electronics.
So, if you take the rest of the definition, I believe that would mean
that in Digital Humanities you'll be researching and analyzing in
electronic form. However, with the few weeks that I've had in class it
seems to be more than that. There is a lot of technological things going
on in Digital Humanities. We've learned how to use different programs in
order to make documents and a different way of writing. Meaning
Microsoft Word has been taking out of this class and replaced by plan
text docs that are written with specific characters that will translate
into different things when converted into different documents. After
reading the Kirschenbaum article, I still didn't quite understand what
exactly DH was. I understand that it's complicated, and it involves
technology and research, but I still didn't quite grasp it.

\subsubsection{Coming to an
Understanding}\label{coming-to-an-understanding}

You would think that having a solid definition, and understanding that
definition would then allow you to understand the topic under
consideration, it didn't. It caused me more confusion. Until, I read
another article by Mark Sample called, \emph{The Digital Humanities is
not About Building, It's About Sharing.} In the article Sample states,
\textgreater{}``The heart of the digital humanities is not the
production of knowledge; it's the \emph{reproduction} of knowledge.''
(Sample) That quote solidified everything to me. Digital Humanities was
about changing the way that knowledge is presented. It's changing the
way that people think about presenting the knowledge that they've
gathered. Three weeks ago I never would have thought I'd be writing in
Markdown, making rtf files in pandoc, or trying to figure out how to
compile an ebook.

\subsubsection{Coming to an End}\label{coming-to-an-end}

Digital Humanities is about using technology to research and teach and
present things in an electronic form. It challenges the typical way of
thinking when it comes to technologically presenting information. There
is more than just Microsoft Word out there.

\subsection{What is DH?}\label{what-is-dh-1}

When it comes to digital humanities everyone seems to draw their own
conclusion as to what they think it is. Some deem it as progressivism,
some view it as building, we go further to say community and sharing,
but what is digital humanities really ? A wise man would inform you that
describing digital humanities, or DH for short would be equivalent to
explaining what water taste like. To begin I will discuss some of the
views of Matthew G. Kirschenbaum who wonders how it has entered our
English departments. This one serves as a more immediate realization of
digital humanities given that I am an English major and had the choice
to take this class an option, also, most of this class has consisted of
a surplus of computer work, thus digital humanities. Kirschenbaum finds
it necessary to refer to wikepedia for the definition and it states
``The digital humanities, also known as humanities computing, is a field
of study, research, teaching, and invention concerned with the
intersection of computing and the disciplines of the humanities''. Now
lets delve a little into what is meant by humanities computing;
University of Chicago believes ``humanities computing, is a field of
study, research, teaching, and invention concerned with the intersection
of computing and the disciplines of the humanities''(Kirschenbaum), but
also use digital humanities as a synonymic two.

Stephen Ramsey is a tenured professor of digital humanities and believes
that it is necessary to know how to ``code'' in digital humanities.
Bluntly Ramsey states, ``Personally, I think Digital Humanities is about
building things. {[}. . .{]} If you are not making anything, you are
not\ldots{}a digital humanist,''(Ramsay) which well conveys connotations
of humanities computing, although he himself may not feel exactly the
same. Coding can be described as ``a form of writing with a dual
audience: machines and other coders (including one's future self)'',
courtesy of digitalhumanitiesnow.org

Julia Flanders explains technology as ``technological progressivism'',
and The narratives that surround technology tend, understandably, to be
progressive.'' The fact that the power of computer chips keep getting
better and faster over the years, as well as an increase in disk utility
shows the advancements of technology, and this advancement and
progression is what we can deem a digital humanity. ``Digital humanities
scholarship to a large degree shares this sense of progress''(Flanders),
Flanders exclaims ``We see, first of all, simple infrastructural
developments that change the social location of computers and bring them
into our sphere of activity.''

In conclusion digital humanities is a hard topic to dissect concisely as
one, but many things amongst the digital and human world
collaboratively. In relation to ``what is digital humanities'',
Kirschenbaum says ``It's tempting to say that whoever asks the question
has not gone looking very hard for an Answer,'' for one will realize the
answer is in multiple perspectives among a similar parallel.

\# Defining the Digital Humanities

The digital humanities are a source of much discussion, controversy and
confusion within humanities departments. While the tools and ideas
present within the digital humanities are several decades old, they are
receiving much more attention in recent years due to their increase in
popularity and the rise of open source software and technology. It is
difficult to get to the heart of what the digital humanities are and why
there is such heated debate over their validity within academia. Many
modern commentators offer answers to that question and the main point of
contention seems to revolve around the availability and sharing of
cultural information that had previously been kept in the few. Digital
humanities are perceived as a threat to literary studies because they
encourage sharing and free access to information and publishing.

One of the major arguments against the digital humanities involves the
sanctity of printed works and their availability. In his article
criticizing the digital humanities, Stephen Marche asserts that the
digitalization of texts and their wide distribution is akin to
disbinding them, and that ``Cutting open the book is literally a return
to the forms and modes of paganism'' (Marche). He also compares the
spread of the humanities via technology to the invention of the printing
press and the subsequent ``early scribal resistance to print'' (Marche),
referring to the scribes that held the proverbial keys to the kingdom on
literacy. This argument is common and underscores the core issue here;
academics and scholars, previously gatekeepers to culture, are less
necessary now that information is readily available.

But what about the digital humanities enables the spread of culture? It
stems from the fact that it is now easier than ever to find, create and
distribute literature and humanities works. Matthew Kirschenbaum sums
this aspect of the subject up perfectly by describing the MLA convention
in which attendees tweeted the happenings of the conference, thus
providing a running commentary and updates in real time (Kirschenbaum).
Mark Sample also analyzes the digital humanities and argues that they
are about sharing and communication. Sample writes, ``We should no
longer be content to make our work public achingly slowly along
ingrained routes, authors and readers alike delayed by innumerable
gateways limiting knowledge production and sharing'' (Sample). In
addition to a new openness in a typically reserved profession, the rise
of single source publishing has played a key role in the production and
distribution of texts. Single source publishing is the creation of a
single text using a specific text format that can then be transformed to
multiple formats using that same original text. Lawyer Ian Sullivan
describes in his article ``Innovation in Practice'' how his law office
prepares Supreme Court briefs using single source publishing tools
available for free online. This is a key example of professional
publishing and document preparation can now be done for free and is
accessible to everyone. The availability of such software and the
ability to share texts so easily in many formats is an example of the
digital humanities is a discipline of sharing and collaboration.

While there are several criticisms of the digital humanities that are
valid, such as the debate over the true essence of the discipline, the
most invective arguments against them seem reactionary and targeted at
the new availability of information. Humanities are no longer confined
to libraries or bookstores, but the internet. Digital scholarship and
techniques allow even major law firms to produce their own custom texts
and share them easily. This places sharing and the flow of information
at the heart of the digital humanities and regardless of the power they
take away from scholars and academics, it also empowers them to explore
the humanities with new ease and efficiency.

\section*{References}\label{references}
\addcontentsline{toc}{section}{References}

Flanders, Julia. ``The Productive Unease of 21st-Century Digital
Scholarship.'' \emph{Digital Humanities Quarterly} 3.3 (2009): n. pag.
Web. 9 July 2014.

Kirschenbaum, Matthew. ``What Is Digital Humanities and What's It Doing
in English Departments?'' \emph{ADE Bulletin} 150.2010 (2010): 1--7.
Web. 27 Oct. 2013.

Marche, Stephen. ``Literature Is Not Data: Against Digital Humanities.''
\emph{The Los Angeles Review of Books} (2014): n. pag. Web. 9 July 2014.

Ramsay, Stephen. ``On Building.'' \emph{Stephen Ramsay} Jan. 2011. Web.
26 Oct. 2013.

Sample, Mark. ``The Digital Humanities Is Not About Building, It's About
Sharing.'' \emph{Sample Reality} May 2011. Web. 26 Oct. 2013.

\end{document}
